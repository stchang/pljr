% compactness.tex
% PL,Jr.,Jr.
% 5/26/2009
% Enderton, Chapter 1.7

\documentclass[12pt]{article}	% YOUR INPUT FILE MUST CONTAIN THESE
\usepackage{url}
\oddsidemargin  -0.5in
\evensidemargin 0.0in
\textwidth      7.5in
\headheight     -1in
\topmargin      0.0in
\textheight     10.0in
\usepackage{amssymb}
\usepackage{amsmath}
\usepackage{mathpartir}

\newcommand{\lub}{\bigsqcup}
\newcommand{\deno}[1]{[\![#1]\!]}
\newcommand{\lift}[1]{\left\lfloor #1 \right\rfloor}


\begin{document}							% TWO LINES PLUS THE \end COMMAND AT
															% THE END

\begin{flushleft}

%%%%%%%%%%%%%%%%%%%%%%%%%%%%%%%%%%%%%%%%%%%%%%%%%%%%%%%%%%%%%%%%%%%%%%%%%%%%%%
%% Definitions
%%%%%%%%%%%%%%%%%%%%%%%%%%%%%%%%%%%%%%%%%%%%%%%%%%%%%%%%%%%%%%%%%%%%%%%%%%%%%%
\textbf{Definitions}\\
A set $\Sigma$ of wffs is \textit{satisfiable} iff there is a truth assignment that satisfies every member of $\Sigma$.\\

\vspace{10mm}

%%%%%%%%%%%%%%%%%%%%%%%%%%%%%%%%%%%%%%%%%%%%%%%%%%%%%%%%%%%%%%%%%%%%%%%%%%%%%%
%% Exercise 1.7.1
%%%%%%%%%%%%%%%%%%%%%%%%%%%%%%%%%%%%%%%%%%%%%%%%%%%%%%%%%%%%%%%%%%%%%%%%%%%%%%
\textbf{Exercise 1.7.1} \\
Assume every finite subset of $\Sigma$ is satisfiable. Show that the same is true of at least one of the sets $\Sigma;\alpha$ and $\Sigma;\neg\alpha$. \\

\vspace{5mm}

\underline{Proof}: \\
Proof by contradiction. Assume that neither every finite subset of $\Sigma;\alpha$ nor every finite subset of $\Sigma;\neg\alpha$ is satisfiable. That means there is some finite $\Sigma_1 \subseteq \Sigma$ such that $\Sigma_1;\alpha$ is unsatisfiable and that there is some finite $\Sigma_2 \subseteq \Sigma$ such that $\Sigma_2;\neg\alpha$ is unsatisfiable. However, $\Sigma_1 \cup \Sigma_2$ is a finite subset of $\Sigma$, so it must be satisfiable, from the stated assumption. This means that there exists some truth assignment $v$ such that $\bar{v}(\varphi) = T$, for $\varphi \in (\Sigma_1 \cup \Sigma_2)$. However, either $\bar{v}(\alpha) = T$ or $\bar{v}(\alpha) = F$, which means that either $\Sigma_1;\alpha$ or $\Sigma_2;\neg\alpha$ is satisfiable. This is a contradiction from what was previously stated. \\

\vspace{10mm}

%%%%%%%%%%%%%%%%%%%%%%%%%%%%%%%%%%%%%%%%%%%%%%%%%%%%%%%%%%%%%%%%%%%%%%%%%%%%%%
%% Exercise 1.7.1
%%%%%%%%%%%%%%%%%%%%%%%%%%%%%%%%%%%%%%%%%%%%%%%%%%%%%%%%%%%%%%%%%%%%%%%%%%%%%%
\textbf{Exercise 1.7.2} \\
Let $\Delta$ be a set of wffs such that (i) every finite subset of $\Delta$ is satisfiable, and (ii) for every wff $\alpha$, either $\alpha \in \Delta$ or $(\neg\alpha) \in \Delta$. Define the truth assignment $v$:

$$
v(A) = 
\begin{cases}
T & \text{iff $A \in \Delta$,}\\
F & \text{iff $A \notin \Delta$}
\end{cases}
$$

for each sentence symbol. Show that for every wff $\varphi$, $\bar{v}(\varphi) = T$ iff $\varphi \in \Delta$. \\

\vspace{5mm}

\underline{Proof}: \\
$\Rightarrow$ We need to show that, for every wff $\varphi$, if $\bar{v}(\varphi) = T$, then $\varphi \in \Delta$. \\

\vspace{2mm}

If $\bar{v}(\varphi) = T$, but $\varphi \notin \Delta$, then by (ii), $(\neg\varphi) \in \Delta$. However, $\bar{v}(\neg\varphi) = F$, which means that there is a finite subset of $\Delta$ that is unsatisfiable. However, this cannot be because according to (i), every finite subset of $\Delta$ is satisfiable. Therefore, if $\bar{v}(\varphi) = T$, then $\varphi \in \Delta$. \\

\vspace{5mm}

$\Leftarrow$ We need to show that, for every wff $\varphi$, if $\varphi \in \Delta$, then $\bar{v}(\varphi) = T$. \\

\vspace{2mm}

We will prove this by structural induction on $\varphi$. \\

\vspace{2mm}

\underline{Case} $\varphi = A$ \\
Since $A \in \Delta$, then $v(A) = T$, by definition of $v$. \\

\vspace{2mm}

\underline{Case} $\varphi = \alpha \wedge \beta$ \\
We first show that if $\varphi \in \Delta$, then both $\alpha$ and $\beta$ are in $\Delta$. If $\varphi \in \Delta$ and $\alpha \notin \Delta$, then by (ii), $\neg\alpha \in \Delta$. However, this means that $\{\varphi,\neg\alpha\}$ is a finite subset of $\Delta$. However, $\{\varphi,\neg\alpha\}$ is unsatisfiable because $\varphi$ is true when $\neg\alpha$ is false and vice versa. Therefore, $\{\varphi,\neg\alpha\}$ cannot be a finite subset of $\Delta$ because according to (i), every finite subset of $\Delta$ is satisfiable. Therefore, if $\varphi \in \Delta$, then $\alpha \in \Delta$. We can also make a similar argument for $\beta$ and therefore, if $\varphi \in \Delta$, then $\alpha,\beta \in \Delta$.

\vspace{2mm}

If $\alpha \in \Delta$, then by the induction hypothesis, $\bar{v}(\alpha) = T$. Similarly, $\bar{v}(\beta) = T$. Since $\varphi = \alpha \wedge \beta$, then $\bar{v}(\varphi) = T$. Therefore, if $\varphi \in \Delta$, then $\bar{v}(\varphi) = T$. \\

\vspace{2mm}

\underline{Case} $\varphi = \alpha \vee \beta$ \\
\underline{Case} $\varphi = \alpha \rightarrow \beta$ \\
\underline{Case} $\varphi = \alpha \leftrightarrow \beta$ \\
\underline{Case} $\varphi = \neg\alpha$ \\
A proof strategy that is similar to the one used for the conjunction case can also be used for these cases.

\vspace{2mm}

Therefore, we have shown that if $\varphi \in \Delta$, then $\bar{v}(\varphi) = T$, for all cases.

\vspace{2mm}

Since we have proven both directions, we can conclude that for every wff $\varphi$, $\varphi \in \Delta$ iff $\bar{v}(\varphi) = T$. \\




\vspace{10mm}




%%%%%%%%%%%%%%%%%%%%%%%%%%%%%%%%%%%%%%%%%%%%%%%%%%%%%%%%%%%%%%%%%%%%%%%%%%%%%%
%% Compactness Theorem
%%%%%%%%%%%%%%%%%%%%%%%%%%%%%%%%%%%%%%%%%%%%%%%%%%%%%%%%%%%%%%%%%%%%%%%%%%%%%%
\textbf{Compactness Theorem} \\
A set of wffs is satisfiable iff every finite subset is satisfiable. \\

\vspace{5mm}

\underline{Proof}: \\
$\Rightarrow$ We need to show that if a set of wffs is satisfiable, then every finite subset is satisfiable. \\

\vspace{2mm}

If a set of wffs is satisfiable, then every finite subset of the set is automatically satisfiable. \\

\vspace{5mm}

$\Leftarrow$ We need to show that if every finite subset of a set of wffs is satisfiable, then the set itself is satisfiable. \\

\vspace{2mm}

Say the set in question is called $\Sigma$.

\begin{enumerate}
	\item Enumerate every wff $\alpha_1$, $\alpha_2$, $\ldots$
	\item Let $\Delta_0 = \Sigma$
	\item Let 
	$
	\Delta_{n+1} = 
	\begin{cases}
	\Delta_n;\alpha{n+1}      & \text{if this is finitely satisfiable,}\\
	\Delta_n;\neg\alpha_{n+1} & \text{otherwise.}
	\end{cases}
	$ \\
	From the result of Exercise 1.7.1, we know that every $\Delta_n$ is satisfiable.
	\item Let $\Delta = \bigcup_n\Delta_n$
	\item We know that every finite subset of $\Delta$ is satisfiable because every finite subset of $\Delta$ is also a subset of some $\Delta_n$, which is finitely satisfiable. We also know that for any wff $\alpha$, either $\alpha \in \Delta$ or $\neg\alpha \in \Delta$. Therefore, if we have a truth assignment $v$ such that $v(A) = T$ iff $A \in \Delta$, from Exercise 1.7.2, we know that if $\varphi \in \Delta$, then $\bar{v}(\varphi) = T$. Since every wff in $\Delta$ is satisfied by $v$, then $\Delta$ is satisfiable.
	\item Since $\Sigma \subseteq \Delta$, $v$ must also safisfy $\Sigma$, so therefore $\Sigma$ is satisfiable.
\end{enumerate}

\vspace{10mm}

%%%%%%%%%%%%%%%%%%%%%%%%%%%%%%%%%%%%%%%%%%%%%%%%%%%%%%%%%%%%%%%%%%%%%%%%%%%%%%
%% Corollary
%%%%%%%%%%%%%%%%%%%%%%%%%%%%%%%%%%%%%%%%%%%%%%%%%%%%%%%%%%%%%%%%%%%%%%%%%%%%%%
\textbf{Corollary} \\
If $\Sigma \models \tau$, then there is a finite $\Sigma_0 \subseteq \Sigma$ such that $\Sigma_0 \models \tau$.

\vspace{10mm}

%%%%%%%%%%%%%%%%%%%%%%%%%%%%%%%%%%%%%%%%%%%%%%%%%%%%%%%%%%%%%%%%%%%%%%%%%%%%%%
%% Exercise 1.7.3 - Proof of Compactness Theorem Using Corollary
%%%%%%%%%%%%%%%%%%%%%%%%%%%%%%%%%%%%%%%%%%%%%%%%%%%%%%%%%%%%%%%%%%%%%%%%%%%%%%
\textbf{Exercise 1.7.3 - Proof of Compactness Theorem Using Corollary} \\
$\Rightarrow$ We must show that if a set of wffs is satisfiable, then every finite subset is satisfiable. \\

\vspace{2mm}

If a set of wffs is satisfiable, then every finite subset is automatically satisfiable. \\

\vspace{5mm}

$\Leftarrow$ We must show that if every finite subset of a set of wffs is satisfiable, then the set itself is satisfiable. \\

\vspace{2mm}

Proof by contradiction. Assume that we have a set of wffs $\Sigma$ such that every finite subset of $\Sigma$ is satisfiable, but that $\Sigma$ itself is unsatisfiable. This means that $\Sigma \models \tau$, for any wff $\tau$. According to the corollary, there is a finite subset $\Sigma_0 \subseteq \Sigma$ such that $\Sigma_0 \models \tau$. Let $\tau = \alpha \wedge \neg\alpha$, which is unsatisfiable. This means that there is no $\Sigma_0 \subseteq \Sigma$ such that $\Sigma_0 \models \tau$. We have a contradiction, so therefore, if every finite subset of $\Sigma$ is satisfiable, $\Sigma$ must be satisfiable.

\end{flushleft}
%\cite{}
%\bibliographystyle{acm}
%\bibliography{template}


\end{document} % THE INPUT FILE ENDS LIKE THIS